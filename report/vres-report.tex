\documentclass[a4paper]{IEEEtran}

% Packages

% NMEA spec environment
\newenvironment{nmeaspec}[1]
{
\newcommand{\field}[2]{\texttt{##1} & ##2 \\}
\vspace{0.2cm}
\noindent\texttt{#1}
\vspace{0.2cm}

\noindent Where: \vspace{0.1cm}\\  \noindent
\vspace{0.2cm}
\begin{tabular}{ll}
}
{
\end{tabular}
}

% Title, author, etc.
\title{Autonomous surf life saving device}
\author{Jarod Lam}
\IEEEspecialpapernotice{2018-2019 VRES project at Queensland University of Technology}

\begin{document}
\maketitle

\begin{abstract}
Make boat that saves people
\end{abstract}

\section{Introduction}


\section{System overview}


\section{Communications}
Communications between the SRB and the base station are done using XBee radios. By attaching a pair of XBee modules to the base station computer and the on-board Arduino, a virtual serial connection is effectively created between the two devices.

\subsection{NMEA 0183 protocol}
NMEA 0183 is a communications specification designed to create a standardised serial interface for GPS devices. Every NMEA `sentence' begins with a \texttt{\$} and ends with \texttt{*CS\textbackslash r\textbackslash n}, where \texttt{CS} is a two-digit hexadecimal checksum of the sentence.

A common NMEA sentence type is \texttt{GPRMC}, the GPS recommended minimum. \texttt{GPRMC} sentences are specified as follows: \cite{gpsinfo}

\begin{nmeaspec}{\$GPRMC,<Time>,<Status>,<Lat>,<LatDir>,\\<Lon>,<LonDir>,<Speed>,<Angle>,<Date>,\\<MagVar>,<MagDir>*CS}
\field{<Time>}{UTC timestamp in HHmmss format}
\field{<Status>}{Status \texttt{A}=active, \texttt{V}=void}
\field{<Lat>}{Latitude in ddmm.mmm format}
\field{<LatDir>}{\texttt{N} or \texttt{S} hemisphere}
\field{<Lon>}{Longitude in dddmm.mmm format}
\field{<LonDir>}{\texttt{E} or \texttt{W} hemisphere}
\field{<Speed>}{Ground speed in knots}
\field{<Angle>}{Track angle in degrees from north}
\field{<Date>}{Date in DDMMYY format}
\field{<MagVar>}{Magnetic variation magnitude}
\field{<MagDir>}{Magnetic variation direction}
\end{nmeaspec}

A NMEA sentence parser was written for the SRB to interpret messages from the on-board GPS and extract location information.

\subsection{Proprietary NMEA sentences}
Some advantages of using NMEA sentences are that they are standardised, human-readable, robust, and relatively simple to implement. Specified below is a set of custom NMEA sentence types was created for communication between the SRB and the base station.

\subsubsection{SRBSM - Status Message}
The \texttt{SRBSM} sentence is sent periodically by the SRB to update the base station with status information.

\begin{nmeaspec}{\$SRBSM,<ID>,<State>,<Lat>,<Lon>,<Speed>,\\<Heading>,<BattV>,<FwdPower>,\\<TgtHeading>*CS}
\field{<ID>}{ID of target SRB}
\field{<State>}{\texttt{0}=disabled, \texttt{1}=manual, \texttt{2}=auto}
\field{<Lat>}{Latitude in decimal degrees}
\field{<Lon>}{Longitude in decimal degrees}
\field{<Speed>}{Speed in metres per second}
\field{<Heading>}{Compass heading in degrees CW from north}
\field{<BattV>}{Current battery voltage}
\field{<FwdPower>}{Forward power from -100 to 100}
\field{<TgtLat>}{Target latitude in decimal degrees}
\field{<TgtLon>}{Target longitude in decimal degrees}
\field{<TgtHeading>}{Target heading in degrees CW from north}
\end{nmeaspec}

\subsubsection{SRBJS - Joystick}
The \texttt{SRBJS} sentence is sent by the base station for manual control of the SRB.

\begin{nmeaspec}{\$SRBJS,<ID>,<FwdPower>,<TgtHeading>*CS}
\field{<ID>}{ID of target SRB}
\field{<FwdPower>}{Forward power from -100 to 100}
\field{<TgtHeading>}{Target heading in degrees CW from north}
\end{nmeaspec}

\subsubsection{SRBWP - Waypoint}
The \texttt{SRBWP} sentence is sent by the base station to autonomously direct the SRB to a set of coordinates.

\begin{nmeaspec}{\$SRBJS,<ID>,<TgtLat>,<TgtLon>,<TgtHeading>,\\<Power>*CS}
\field{<ID>}{ID of target SRB}
\field{<TgtLat>}{Target latitude in decimal degrees}
\field{<TgtLon>}{Target longitude in decimal degrees}
\field{<TgtHeading>}{Target heading in degrees CW from north}
\field{<Power>}{Motor power to use from 0-100}
\end{nmeaspec}

\bibliography{bibliography}
\bibliographystyle{IEEEtran}

\end{document}